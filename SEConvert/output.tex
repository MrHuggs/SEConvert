
\documentclass[a4paper]{article}

%% Language and font encodings
\usepackage[english]{babel}
\usepackage[utf8x]{inputenc}
\usepackage[T1]{fontenc}

%% Sets page size and margins
\usepackage[a4paper,top=3cm,bottom=2cm,left=.5cm,right=.5cm,marginparwidth=1.75cm]{geometry}

%% Useful packages
\usepackage{amsmath}
\usepackage{amssymb}
\usepackage{graphicx}
\usepackage[colorinlistoftodos]{todonotes}
\usepackage[colorlinks=true, allcolors=blue]{hyperref}
\setlength{\parskip}{1em}
$\def\ip {\, \lrcorner \, }\def\Ab {\,\text{Ab} \,}\def\Hom {\,\text{Hom} \,} \def\im{\,\text{im} \,}\def\h#1{H^#1_{dR}}$
\section{Statement}

If $G$ is a group, the \section{commutator subgroup of $G$,} denoted $[G,G]$ is the smallest normal subgroup containing all elements of the form $g_1 g_2 g_1^{-1}g_2^{-1}$ for $g_1, g_2 \in G;$ and the \section{abelianization of G},  denoted by $\Ab(G)$ is the quotient group $G / [G,G].$

Suppose $M$ is a connected smooth manifold and $q \in M.$ It can be shown that there is a group homomorphism from $\pi_1(M, q)$ to $H_1(M)$ that sends the homotopy class of a loop $\gamma$ to the homology class of the $1$-cycle determined  by $\gamma,$ and this map descends to an isomorphism from $\Ab(\pi_1(M,q))$ to $H_1(M).$

Use this result together with the de Rham Theorem to prove that the map $\Phi : H_{dR}^1(M) \rightarrow \Hom (\pi_1(M,q), \Bbb R)$ of Theorem 17.17 is an isomorphism.

\section{Theorem 17.17 (First Cohomology and the Fundamental Group)}

Suppose $M$ is a connected smooth manifold. For each $q\in M$, the linear map $\Phi : H_{dR}^1(M) \rightarrow \Hom(\pi_1(M, q), \Bbb R)$ is well defined and injective.

The map is:

\begin{align}{\cal I} [\omega][\gamma] = \int_\gamma \omega.
\end{align}

\section{The de Rham Theorem} 

The \section{de Rham Theorem} (18.14) states that:
\begin{align}{\cal I} : H^p_{dR}M \rightarrow H^p(M; \Bbb R) \cong \Hom (H_p(M); \Bbb R)
\end{align}
is an isomorphism. 

\section{Proof}

Since 17.17 shows that $\Phi$ is well defined and injective, what remains to be shown is that it is surjective.

We have $\Ab(\pi_1(M,q))  \cong H_1(M).$

Therefore:
\begin{align}\begin{align}H_{dR}^1(M) &\cong \Hom (H_1(M); \Bbb R)\\& \cong  \Hom (\Ab(\pi_1(M,q)); \Bbb R).\end{align}
\end{align}

Finally, from \section{Abelianization and Hom,} we have:
\begin{align}\Hom (\Ab(\pi_1(M,q); \Bbb R) \cong \Hom (\pi_1(M,q); \Bbb R)
\end{align}

And then finally:

\begin{align}H_{dR}^1(M) \cong \Hom (\pi_1(M,q); \Bbb R).\, \blacksquare
\end{align}\end{document}